\documentclass{acm_proc_article-sp}[10pt]
\usepackage{cite}
\usepackage[english]{babel}
\usepackage{listings}
\usepackage{subfigure}
\usepackage{url}
\lstset{language=C,
       basicstyle=\small,
       numbers=left,
       stepnumber=1,
       numbersep=5pt,
       commentstyle=\color{red},
       escapeinside={(*@}{@*)}}

%%%%%%%%%%%%%%%
% Commenting on the paper:
%%%%%%%%%%%%%%%%%
\definecolor{midblue}{rgb}{0.7,0.9,1}
\definecolor{orange}{rgb}{1,0.9,0.6}
\definecolor{lightgreen}{rgb}{0.6,1,0.6}
\definecolor{grey}{rgb}{0.9,0.9,0.9}

\newcommand{\comment}[2]{\begin{center}\colorbox{#1}{\parbox{0.85\linewidth}{\textit{{#2}}}}\end{center}}

\newcommand{\abdullah}[1]{\comment{orange}{{Abdullah: #1}}}
\newcommand{\lauro}[1]{\comment{midblue}{{Lauro: #1}}}
\newcommand{\elizeu}[1]{\comment{lightgreen}{{Elizeu: #1}}}
\newcommand{\matei}[1]{\comment{grey}{{Matei: #1}}}

%\title{Totem: Massively-Parallel Graph Processing Framework}
\title{On Scheduling Graph Workloads on Hybrid Massively-Parallel Platforms}
\subtitle{(or The Graph Workload Partitioning Manifesto)}
\numberofauthors{1}
\author{
% 1st. author
\alignauthor 
Authors
%Lauro Beltr\~ao Costa, Abdullah Gharaibeh, Elizeu
%Santos-Neto\vspace{3mm}\\
%       \affaddr{\small{University of British Columbia}}\\
%       \affaddr{\small{2332 Main Mall, Vancouver, BC, CANADA}}\\
%       \email{\small{\{lauroc,abdullah,elizeus\}@ece.ubc.ca}}
}
\date{November 2011}

\begin{document}

\maketitle

%\abdullah{Comments by Abdullah}
%\lauro{Comments by Lauro}
%\elizeu{Comments by Elizeu}

\begin{abstract}
Graphs have recently attracted the attention of the research and industry communities alike. In particular, the focus has been on how to leverage massively parallel architectures to speed up graph algorithms. Although harnessing architecture-specific characteristics is likely to boost graph algorithms performance on parallel machines, little attention has been devoted to graph workload aspects and their relationship to the performance of particular components in a hybrid parallel processing platform (e.g., CPU-GPU clusters). More importantly, as the workloads become larger than the capacity of shared memory machines, it will be necessary to efficiently partition the graphs such that it maximizes the processing throughput. This draft proposes a performance model to help reasoning about graph workload partitioning on massively parallel platforms composed of heterogeneous processors and accelerators such as {\sc gpu}s.
\end{abstract}

% A category with the (minimum) three required fields
%\category{H.4}{Information Systems Applications}{Miscellaneous}
%A category including the fourth, optional field follows...
%\category{D.2.8}{Software Engineering}{Metrics}[complexity measures, performance measures]
%\terms{Theory}
%\keywords{ACM proceedings, \LaTeX, text tagging} % NOT required for Proceedings

\section{Introduction}
Graphs are everywhere. From the nowadays omnipresent online social networking tools to the structure of fundamental scientific problems, passing through mundane applications such as finding the best bicycle route between two points in a city, graphs are indeed a building block to model several important problems.

However, the scale of current graphs calls for new computational platforms and techniques that support efficient processing. Currently, two main platforms are common: on the one end, high-performance, yet expensive and specialized supercomputers such as Cray machines have been deployed~\cite{mizell2009early, yoo2005scalable}; on the other end, less efficient, yet commodity and expandable traditional clusters are also being used in production systems~\cite{Malewicz2009}. Between these two ends, hybrid commodity platforms (e.g., GPU-supported clusters) have the potential to offer the good of two platforms: a high-performance, low-cost system.

To this end, this work investigates the general challenges of processing graph algorithms on hybrid commodity systems (e.g., GPU-supported clusters). Additionally, in the spirit of building abstractions to hide complexity, the ultimate goal is to build a generic graph-processing framework that leverages massively-parallel hybrid platforms. In particular, this work aims at addressing the following questions:

\begin{enumerate}
\item What are the general challenges to support graph processing on a hybrid, GPU-based compute node? Is it possible to partition the processing to efficiently use both the main processor and the GPU? In other words, given a fixed die area or power budget, is it more efficient to rely on a traditional symmetric multi-processor (SMP) or a hybrid node?

\item Assuming that a hybrid, GPU-based node brings performance benefits, is it efficient to use a GPU-supported cluster to process large-scale graph problems compared to a traditional cluster? In other words, in a traditional cluster setup, is the computation phase dominant enough compared to the communication phase such that adding a massively-parallel component to speedup the computation phase would improve the end-to-end system performance?

\item Assuming that graph algorithms can efficiently leverage GPU-supported clusters, what should a graph processing framework that aims at simplifying the task of implementing graph algorithms on such platforms provide to developers? Specifically, what is the adequate parallel processing model (e.g., BSP - Bulk Synchronous Processing~\cite{Valiant1990}, PRAM~\cite{Fortune78}, or LogP~\cite{Culler1996}), graph-level abstractions (e.g., vertex or edge centric), and the high-level interface of a sufficiently expressive, yet performance-efficient framework?

\item Hiding complexity and problem-specific optimizations are often conflicting goals: a framework that reduces complexity may mask hardware details that could be used to optimize specific graph algorithms. With this in mind, what is the performance cost generated by introducing an abstraction layer between the graph algorithms and the hardware platform?

\item Ultimately, a graph-processing framework should enable measurable gains from the application standpoint. Thus, can we identify applications (e.g., PageRank) where the use of such framework enables problem-solving at larger scales or faster than current platforms?

\end{enumerate}

At this stage, we report on our progress towards addressing the first question. We present preliminary performance results for a number of graph algorithms on a single GPU and compare them with high-end SMP systems. We show that graph algorithms are memory-latency bound, and that traditional SMP systems are not optimized to support such a processing pattern. We also show that a GPU can offer important performance speedups due to its ability to hide memory access latency through massive multithreading.

The next sections present the opportunities, goals, and anticipated challenges of this project (\S~\ref{sec:motivation}); related work (\S~\ref{sec:related}); methodology (\S~\ref{sec:methodology}); current progress (\S~\ref{sec:progress}); future work (\S~\ref{sec:future}); and, concluding remarks (\S ~\ref{sec:conclusion}).

\section{Problem}
\label{sec:problem}

Given a set of processors $\mathcal{P}$ and a workload $\mathcal{W}$, the {\em graph workload partitioning} problem of maximizing Equation~\ref{eq:throughput_sys} is equivalent to the following problem:

\begin{equation*}
  \begin{aligned}
   \underset{i}{\text{maximize }}
    & t_i(\mathcal{W}_i) \\
    \text{subject to}
    & & |\mathcal{W}_i| \leq |p_i|, \; i = 1, \ldots, c.
  \end{aligned}
\end{equation*}

where $|\mathcal{W}_i|$ denotes the memory footprint of a graph workload partition $\mathcal{W}_i$, and $|p_i|$ indicates the memory capacity of processor $p_i$. 

Assuming that the time a processor takes to process a given edge is constant, if the edge's vertices are located in the processor's local memory, the problem can be written as a minimization of the number of edges in each partition that have remote vertices. More formally,   

\begin{equation*}
  \begin{aligned}
    \text{minimize }
    & \sum_i^c|E_i^r| \\
    \text{subject to}
    & & |\mathcal{W}_i| \leq |p_i|, \; i = 1, \ldots, c.
  \end{aligned}
\end{equation*}

\abdullah{even if we assume homogenous processors, minimizing $|E_i^r|$ is not good enough to solve the partitioning problem, we need to make the partitions balanced in terms of amount of work (e.g., one can have a cut with zero $|E_i^r|$ by having the whole graph processed on one processor)}

By making some assumptions about $T_i(v)$ (i.e., the time $p_i$ takes to process a vertex), we could translate this problem further into a {\em modularity maximization} problem and start with informed graph partitioning algorithms based on Newman's approach~\cite{Newman2006}.

\elizeu{{\sc ToDo}: \\
        \begin{enumerate}
          \item Check/improve notation; 
          \item Prove that this problem is NP-hard by showing it is exactly as a graph partitioning problem with constraints;
          \item List and discuss promising heuristics; 
          \item Extend the model to compute the partitioning algorithm cost in the total application execution -- the idea is to fold this into the workload $\mathcal{W}^{*} = (\mathcal{A}^{*}, G)$ submitted to $p_0$, where $\mathcal{A}^{*}$ is a particular graph partitioning algorithm;
          \item Characterize Equation~\ref{eq:throughput_p} with some experiments on a CPU/GPU environment.
        \end{enumerate}}

Lets assume a system with a heterogeneous set of processors. A simple example would be a system with one CPU and one GPU. In such a system, a number of factors affect how good or bad one processor unit performs with respect to another for a specific algorithm. Factors such as the processing unit's characteristics (e.g., number of hardware threads, caches), the partition characteristics (e.g., density, edge distribution) and the algorithm's implementation (e.g., how many cache misses it incur).

One idea to estimate the throughput of a processor is to use linear regression. The idea is to perform controlled experiments that varies the graph features for all types of processors and algorithm implementations. The results will be used as a training set to feed the linear regression model.

\section{Related Work}
\label{sec:related}

\elizeu{This section is the same as in the 571R final report.}

This project spans the following areas: graph algorithms, in general, and parallel graph algorithms, in particular; graph algorithms on GPUs; and parallel graph processing frameworks. This section briefly positions this work among the related literature.

{\bf Parallel graph algorithms.} Although graph algorithms are a well studied area, leveraging parallel architecture to accelerate the algorithm runtime is far from a straightforward task. A parallel implementation may require substantial changes to the original sequential algorithm and strong assumptions about the parallel platform. The required changes and assumptions come at the expense of making the algorithm less portable across platforms. Sometimes, graph problems that have optimal sequential algorithms, dlack a optimal parallel counterpart (e.g., Dijkstra's algorithm is a provably optimal sequential algorithm for the single source shortest path, but an optimal parallel algorithm for this problem is unknown).  

Although the PRAM~\cite{Fortune78} parallel processing model is vastly used in the study of parallel algorithms~\cite{Quinn1984,Atallah1984}, some recent works try to provide performance analysis on distributed memory machines models. Meyer and Sanders~\cite{Meyer2003}, for example, gives a parallel algorithm for single source shortest path named $\Delta$-stepping. The authors analyze the performance of the algorithm in the PRAM model and provide extensions to distributed memory model. The algorithm works by keeping nodes with tentative distances in separate buckets where each bucket represent the distances within the range of size $\Delta$. In each phase, the algorithm consider the nodes in the non-empty buckets and edges of weight up to $\Delta$. Parallelism is achieved by processing the nodes in a given bucket concurrently. 

{\bf Graph algorithms on GPUs.} Previous work investigates the use of GPUs to accelerate graph computation~\cite{Harish2007, Katz2008, Sungpack2010, dehne2010exploring}. For example, Sungpack et al.~\cite{Sungpack2010} shows that, in the case of breadth-first search, a single Nvidia Tesla GPU offers up to 2.5x speedup compared to dual socket, quad-core Intel Xeon symmetric multiprocessor on realistic workloads. However, those works focus on few graph algorithms, mainly breadth-first search, single source shortest path, and all pairs shortest path. Moreover, the implementations assume that the graph fits GPU memory, which puts a limitation on the size of graphs that can be processed. 

A natural scaling path is to run large graph algorithms into multi-GPU systems. Katz et al.~\cite{Katz2008} implement a hand-crafted version of all pairs shortest path algorithm for multi-GPU systems. However, the proposed approach is tightly coupled with the problem and cannot be generalized to other algorithms. Additionally, the implementation provided by Katz et al. assumes an adjacency matrix graph representation that imposes a significant space overhead for sparse graphs, a characteristic that is common in the majority of real-world graphs. Yet another, nevertheless fundamental, challenge to harness multi-GPU systems is to adapt these algorithms from the PRAM model to a distributed memory machines model such as BSP or LogP. Gerbessiotis and Valiant provide results on the performance impact on emulating a DMM model with PRAM~\cite{Gerbessiotis92}. 

{\bf Graph processing frameworks.} The Parallel BGL~\cite{gregor2005parallel} and Pregel~\cite{Malewicz2009} present frameworks to implement distributed graph algorithms. Both frameworks specify generic concepts for designing distributed graph algorithms. The frameworks are based on abstractions that are common among graph algorithms. Examples of such abstractions include vertices, edges, property maps {\em (key-value pairs)}, and mechanisms for data propagation and graph traversal. 

It is worth noting that PBGL, Pregel, and others~\cite{Zhao2009} assume traditional cluster systems (i.e., nodes connected via network and containing traditional multi-core processors). This work targets, on the other hand, hybrid GPU-based systems, which have different computational model and offer different trade-offs and challenges. 

To the best of our knowledge, there is no graph processing framework optimized for hybrid GPU-based platforms. A framework specialized for GPU-based platforms could minimize communication overhead over the GPU's high-latency I/O channels (e.g., employing compact messaging and graph representations to minimize communication and space footprint); also it could offer graph-specific abstractions that hide the GPU's complex memory model while efficiently leveraging it (e.g., transparently utilizing shared memory). 

\section{Concluding Remarks}
\label{sec:conclusion}



%% References
\bibliographystyle{abbrv}
\bibliography{graphs}
\balancecolumns
\end{document}
