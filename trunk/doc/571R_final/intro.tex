\section{Introduction}
Graphs are everywhere. From the nowadays omnipresent online social networking tools to the structure of fundamental scientific problems, passing through mundane applications such as finding the best bicycle route between two points in a city, graphs are indeed a building block to model several important problems.

However, the scale of current graphs calls for new computational platforms and techniques that support efficient processing. Currently, two main platforms are common: on the one end, high-performance, yet expensive and specialized supercomputers such as Cray machines have been deployed ~\cite{mizell2009early, yoo2005scalable}; on the other end, less efficient, yet commodity and expandable traditional clusters are also being used in production systems ~\cite{Malewicz2009}. Between these two ends, hybrid commodity platforms (e.g., GPU-supported clusters) have the potential to offer the good of two platforms: a high-performance, low-cost system.

To this end, this work investigates the general challenges of processing graph algorithms on hybrid commodity systems (e.g., GPU-supported clusters). Additionally, in the spirit of building abstractions to hide complexity, the ultimate goal is to build a generic graph-processing framework that leverages massively-parallel hybrid platforms. In particular, this work aims at addressing the following questions:

\begin{enumerate}
\item What are the general challenges to support graph processing on a hybrid, GPU-based compute node? Is it possible to partition the processing to efficiently use both the main processor and the GPU? In other words, given a fixed area or power budget, is it more efficient to rely on a traditional multi-core or a hybrid node?

\item Assuming that in hybrid, GPU-based node brings performance benefits, is it efficient to use a GPU-supported cluster to process large-scale graph problems compared to a traditional cluster? In other words, in a traditional cluster setup, is the computation phase dominant enough compared to the communication phase so that adding a GPU to speedup the computation phase would improve the end-to-end performance?

\item Assuming that graph algorithms can efficiently leverage GPU-supported clusters, what should a graph processing framework that aims at simplifying the task of implementing graph algorithms on such platforms provide to developers? Specifically, what is the adequate parallel processing model (e.g., BSP - Bulk Synchronous Processing~\cite{Valiant1990}, PRAM~\cite{Fortune78}, or LogP~\cite{Culler1996}), graph-level abstractions (e.g., vertex or edge centric), and the high-level interface of a sufficiently expressive, yet performance-efficient framework?

\item Hiding complexity and problem-specific optimizations are often conflicting goals: a framework that reduces complexity may mask hardware details that could be used to optimize specific graph algorithms. With this in mind, what is the performance cost, if any, generated by introducing an abstraction layer between the graph algorithms and the hardware platform?

\item Ultimately, a graph-processing framework should enable measurable gains from the application standpoint. Thus, can we identify applications (e.g., PageRank) where the use of such framework enables problem-solving at larger scales or faster than current platforms?

\end{enumerate}

The next sections present the opportunities, goals, and anticipated challenges of this project (\S~\ref{sec:motivation}); related work (\S~\ref{sec:related}); methodology (\S~\ref{sec:methodology}); current progress (\S~\ref{sec:progress}); future work (\S~\ref{sec:future}); and, concluding remarks (\S ~\ref{sec:conclusion}).
