\section{Future Work}
\label{sec:future}

Implementing {\sc bfs}, Dijkstra's and PageRank algorithms already helped to identify the commonalities among graph algorithms. For example, operations that sweep through the list of neighbors of each vertex are present in all three algorithms. Therefore, defining common building-block kernels that can be used by a variety of graph algorithms is clearly a potential way to go. Additionally, and equally important, there are logistics operations such as graph parsing and initialization, memory allocation, and data transfer routines that can also be part of an infrastructure shared by all the algorithms. In fact, some of these operations are already provided as abstractions in the current version of {\sc totem}.

Besides the abstractions that are undoubtedly useful in single-GPU hybrid platforms, it is also necessary to study and design new abstractions for a multi-GPU environment. To this end, we will first investigate the potential losses in performance due to the mismatch between the parallel programming model assumed by the algorithms (i.e., PRAM) and the model that best match a multi-GPU platform (e.g., BSP). This effort can benefit from previous studies on the translation of parallel algorithms from on model to another~\cite{Gerbessiotis92}. 

It is also important to show benefits at the application layer. Therefore, we consider a list of candidate applications that could provide evidence of {\sc Totem}'s applicability and good estimates of the application-level performance that {\sc Totem} can deliver. 

{\bf Community detection.} Online social systems are commonplace nowadays. Graphs are unsurprisingly the mathematical abstraction of choice to study the characteristics of user interactions in these systems. One important application is the detection of subgroups of users where the connections are stronger within the group than those to individuals outside the group. This application is called {\em community detection}. The scale of existing social networks demands efficient and scalable solutions for community detection. This application is a good candidate as it may build upon the already implemented algorithms (e.g., Girwan-Newman's community detection algorithm~\cite{Newman2004} which is based on a centrality measure that uses {\sc bfs}).

{\bf Characterization of time-evolving social networks.} Although online social networks are inherently dynamic (i.e., users are actively exchanging public/private messages, producing and consuming content), the vast majority of studies that analyze such network focus on static snapshots of the network~\cite{Willinger2009}, specially due to the computational cost of analysing large networks while considering their time-evolving characteristics. Thus, we plan to use {\sc Totem} to enable the characterization of structural aspects of dynamic online social networks. In particular, we focus on the following question: what is the rate of variation in node centrality measures in a network with time-evolving edge weights? This study can also leverage already implemented algorithms by extending them to compute centrality measures efficiently.

Finally, the initial workload characterization points to an interesting vein of work that investigates what graph characteristics such as its structure can help predicting the attainable performance of graph algorithms on these hybrid platform. Also, in a more advance phase, these correlations can be explored in the design of scheduling policies that redirect the graph processing to the most adequate processor or accelarator depending on the graph characteristics (e.g., node degree distribution).
