\section{Introduction}
Graphs are everywhere. From the nowadays omnipresent online social networking tools to the structure of fundamental scientific problems, passing through mundane applications such as finding the best bicycle route between two points in a city, graphs are indeed a building block to model several important problems.

Although graphs are powerful abstractions and algorithms that help tackling complex problems, the scale of current graphs calls for new computational platforms and techniques that can lead to efficient processing. In this spirit, previous works investigate the use of massively-parallel processors, particularly GPUs, to accelerate graph computation~\cite{Harish2007, Katz2008, Sungpack2010, dehne2010exploring}. However, the general challenges of porting algorithms for large-scale graph processing to GPU-based platforms (which are turning into mainstream), especially large multi-GPU systems, are vastly unknown. This work aims at filling this gap by addressing the following questions:

\begin{enumerate}
\item What are the general challenges in porting algorithms for large scale graph processing to hybrid GPU-based systems? Can the computational resources offered by such systems be efficiently harnessed by graph algorithms? In other words, given a fixed area or power budget, is it more efficient to rely on traditional multi-core or hybrid systems?

\item Assuming that graph algorithms can efficiently leverage GPU-based platforms, what should a graph processing framework that aim at simplifying the task of implementing graph algorithms on such platforms provide to developers? Specifically, what is the most adequate parallel processing model (e.g., BSP - Bulk Synchronous Processing~\cite{Valiant1990}, PRAM~\cite{Fortune78}, or LogP~\cite{Culler1996}), the abstractions (e.g., vertex or edge centric), and the high-level interface of a sufficiently expressive, yet performance efficient framework? 

\item Hiding complexity and problem-specific optimization are often conflicting goals, as a framework that reduces complexity may also mask some hardware details that could have been used to better optimize specific graph algorithms. With this in mind, what would be the performance cost, if any, due to the introduction of a framework layer between the graph algorithms and the hardware platform? 

\item Ultimately, a graph-processing framework should enable measurable gains from the application stand point. Thus, can we identify applications where the use of such framework enables problem-solving at larger scales than current platforms? For example, could one design more efficient algorithms for community detection in dynamic networks using the abstractions provided by the framework?

\end{enumerate}

To address these questions, this work follows a top-down approach: it studies a number of core graph algorithms and graph-based applications to inform the design of the framework.
The next sections present the opportunities, goals, and anticipated challenges of this project (\S~\ref{sec:opp}); related work (\S~\ref{sec:related}); methodology (\S~\ref{sec:methodology}); current progress (\S~\ref{sec:progress}); future work (\S~\ref{sec:future}); and, concluding remarks (\S ~\ref{sec:conclusion}).

\section{Opportunities and Challenges}
\label{sec:opp}

This project is motivated by two observations. First, current GPUs offer significant peak performance advantages (computation and internal memory bandwidth) compared to traditional multiprocessors. Second, there is a lack of efficient and simple graph processing frameworks tailored to harness massively parallel platforms such as GPUs. 

In particular, large-scale graphs are data intensive in nature: graph applications in domains such as social networks and web analysis process billions of vertices with little work per vertex, which suggests an opportunity for high data parallelism. Therefore, these characteristics make such applications suitable for the new generation of massively-parallel computing architectures such as hybrid GPU-based platforms.

To realize such opportunities, however, it is necessary to overcome some challenges. First, the large amount of data required by large-scale graphs puts great pressure on two scarce GPU resources: on-board memory and host-device I/O bandwidth. Second, graph problems generally exhibit irregular parallelism~\cite{Kulkarni2009}, and have little locality on memory access patterns (e.g., vertex neighbours may be scattered in memory). Finally, communication between compute resources in a heterogeneous multi-GPU system may lead to performance bottlenecks due to irregular parallelism. These characteristics make it challenging to exploit GPU architectures, which have strict parallel compute model (i.e., Single Instruction Multiple Data), and rely on regular memory access patterns to achieve good performance.

Considering the opportunities and specific challenges above, the specific goals of this project are:

\begin{enumerate}
\item To understand the challenges of porting graph algorithms to GPUs. Example problems include breadth-first search, shortest path algorithms, PageRank, semi-clustering and bipartite matching.

\item To design a preliminary version of a graph computing framework for GPU-based platforms. The framework aims to provide an infrastructure to simplify the task of implementing parallel graph algorithms in large-scale GPU-based systems (e.g., multi-GPU systems and GPU clusters).

\item To implement and evaluate a number of use case applications. In particular, we are interested in efficient community detection in large-scale social networks, and characterizing the evolution of centrality measures in time-evolving online social networks. 
\end{enumerate}
