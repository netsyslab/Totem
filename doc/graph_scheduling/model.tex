\documentclass{acm_proc_article-sp}[10pt]
\usepackage{cite}
\usepackage[english]{babel}
\usepackage{listings}
\usepackage{subfigure}
\usepackage{url}
\lstset{language=C,
       basicstyle=\small,
       numbers=left,
       stepnumber=1,
       numbersep=5pt,
       commentstyle=\color{red},
       escapeinside={(*@}{@*)}}

%%%%%%%%%%%%%%%
% Commenting on the paper:
%%%%%%%%%%%%%%%%%
\definecolor{midblue}{rgb}{0.7,0.9,1}
\definecolor{orange}{rgb}{1,0.9,0.6}
\definecolor{lightgreen}{rgb}{0.6,1,0.6}
\definecolor{grey}{rgb}{0.9,0.9,0.9}

\newcommand{\comment}[2]{\begin{center}\colorbox{#1}{\parbox{0.85\linewidth}{\textit{{#2}}}}\end{center}}

\newcommand{\abdullah}[1]{\comment{orange}{{Abdullah: #1}}}
\newcommand{\lauro}[1]{\comment{midblue}{{Lauro: #1}}}
\newcommand{\elizeu}[1]{\comment{lightgreen}{{Elizeu: #1}}}
\newcommand{\matei}[1]{\comment{grey}{{Matei: #1}}}

%\title{Totem: Massively-Parallel Graph Processing Framework}
\title{On Scheduling Graph Workloads on Hybrid Massively-Parallel Platforms}
\subtitle{(or The Graph Workload Partitioning Manifesto)}
\numberofauthors{1}
\author{
% 1st. author
\alignauthor 
Authors
%Lauro Beltr\~ao Costa, Abdullah Gharaibeh, Elizeu
%Santos-Neto\vspace{3mm}\\
%       \affaddr{\small{University of British Columbia}}\\
%       \affaddr{\small{2332 Main Mall, Vancouver, BC, CANADA}}\\
%       \email{\small{\{lauroc,abdullah,elizeus\}@ece.ubc.ca}}
}
\date{November 2011}

\begin{document}

\maketitle

%\abdullah{Comments by Abdullah}
%\lauro{Comments by Lauro}
%\elizeu{Comments by Elizeu}

\begin{abstract}
Graphs have recently attracted the attention of the research and industry communities alike. In particular, the focus has been on how to leverage massively parallel architectures to speed up graph algorithms. Although harnessing architecture-specific characteristics is likely to boost graph algorithms performance on parallel machines, little attention has been devoted to graph workload aspects and their relationship to the performance of particular components in a hybrid parallel processing platform (e.g., CPU-GPU clusters). More importantly, as the workloads become larger than the capacity of shared memory machines, it will be necessary to efficiently partition the graphs such that it maximizes the processing throughput. This draft proposes a formal system model to help reasoning about graph workload partitioning on massively parallel platforms composed of heterogeneous processors and accelerators such as {\sc gpu}s.
\end{abstract}

% A category with the (minimum) three required fields
%\category{H.4}{Information Systems Applications}{Miscellaneous}
%A category including the fourth, optional field follows...
%\category{D.2.8}{Software Engineering}{Metrics}[complexity measures, performance measures]
%\terms{Theory}
%\keywords{ACM proceedings, \LaTeX, text tagging} % NOT required for Proceedings

\section{Introduction}
\label{sec:intro}
Let us consider a collection of processors $\mathcal{P} = \{p_0\} \cup \{p_1,\ldots,p_c\}$, where $p_i$ denotes a processor, and $c$ indicates the number of processors in the sytem, while $p_0$ is a special processor that has the sole responsibility of distributing the workload and gathering the results, unless stated otherwise all the dicussion focuses on processors $p_i$ for $i>0$. Every two processors $p_i,p_j \in \mathcal{P}$ are connected by a set of bidirectional links $\mathcal{L} = \{l_{ij}\}$, where $i \neq j$, and $l_{ij} = l_{ji}$. Currently, we assume the bandwidth is homogeneous across links and it is represented by $B$.

The platform of interest is composed of processors with varying processing speeds. Moreover, the speed of a given processor $p_i$ is a function of the workload submited to it. To account for this system characteristic, we represent the processing throughput of a processor $p_i$ by a function $t_i(\mathcal{W}_k)$, where $\mathcal{W}_k$ is a partition of the total workload $\mathcal{W}$ submitted to the system. 

A workload is represented by a pair ({\em algorithm, graph}) denoted by $\mathcal{W} = (\mathcal{A}, G)$, where $\mathcal{A}$ is any relevant algorithm such as {\sc bfs}, and $G$ is a graph, defined as follows: let $G = (V, E, w)$ be weighted directed graph, where, as usual, $V = \{v_i, \ldots,v_n\}$ is the set of vertices, $E$ is the set of directed edges (i.e., $(v_i,v_j) \neq (v_j, v_i)$ for $i \neq j$, and $w: E \mapsto \mathbb{R}$ is a weight function. Without loss of generality, we assume that $|V| = n$ and $|E| = m$.

Therefore, a workload partition $\mathcal{W}_k$ is, in practical terms, a pair with an algorithm $\mathcal{A}$ and a graph partition $G_k = (V_k, E_k) \subseteq G$. It is worth noting that edges in $E_k$ may have source and destination vertices that are not in $V_k$. This assumption impacts the performance of a processor under a given workload, as the algorithm may need to access vertices located at a different processor via the interconnection link. More formally, the throughput function of processor $p_i$ under a workload partition $\mathcal{W}_k$ is given by:

\begin{equation}
t_i(\mathcal{W}_k) = \frac{|E_k|}{\sum_{e \in E_k}T_i(e) + |E_k^{r}|B}
\end{equation}\label{eq:throughput_p}

where $T_i(e)$ is the time processor $p_i$ takes to process an edge $e$ when the edge is located in its local memory, and $E_k^{r} = \{(v_i,v_j) | v_i \vee v_j \notin V_k\} \subseteq E_k$ represents the subset of edges s.t. either the source or destination vertice is not located in $p_i$'s local memory. Intuitively, the less communication a processor needs to access the edges in the graph partition assigned to it, the higher is the achievable throughput.

Assuming that the workload $\mathcal{W}$ size is smaller than, or equal to, the aggregated memory available in each processor, we define the throughput of a system $\mathcal{P}$, as follows: 

\begin{equation}
t(\mathcal{P}, \mathcal{W}) = \min_{i}\left\{t_i(\mathcal{W}_i\right\}, i = {1, \ldots, c}
\label{eq:throughput_sys}
\end{equation}

The intuition behind Equation~\ref{eq:throughput_sys} is that the maximum performance of a parallel system is always limited by its slowest component. In the scenarios we investigate, the performance of a component is limited by the worload partition assigned to it by the master node. As described in the next section, the problem of partioning the graph is can be framed as a global system optimization problem that aim to minimize the amount of communication needed across processors for a given graph workload.



\section{Problem}
\label{sec:problem}

Given a set of processors $\mathcal{P}$ and a workload $\mathcal{W}$, the {\em graph workload partitioning} problem of maximizing Equation~\ref{eq:throughput_sys} is equivalent to the following problem:

\begin{equation*}
  \begin{aligned}
   \underset{i}{\text{maximize }}
    & t_i(\mathcal{W}_i) \\
    \text{subject to}
    & & |\mathcal{W}_i| \leq |p_i|, \; i = 1, \ldots, c.
  \end{aligned}
\end{equation*}

where $|\mathcal{W}_i|$ denotes the memory footprint of a graph workload partition $\mathcal{W}_i$, and $|p_i|$ indicates the memory capacity of processor $p_i$. 

Assuming that the time a processor takes to process a given edge is constant, if the edge's vertices are located in the processor's local memory, the problem can be written as a minimization of the number of edges in each partition that have remote vertices. More formally,   

\begin{equation*}
  \begin{aligned}
    \text{minimize }
    & \sum_i^c|E_i^r| \\
    \text{subject to}
    & & |\mathcal{W}_i| \leq |p_i|, \; i = 1, \ldots, c.
  \end{aligned}
\end{equation*}

By making some assumptions about $T_i(e)$ (i.e., the time $p_i$ takes to process an edge), we could translate this problem further into a {\em modularity maximization} problem and start with informed graph partioning algorithms based on Newman's approach~\cite{Newman2006} and .

\elizeu{{\sc ToDo}: \\
        \begin{enumerate}
          \item Check/improve notation; 
          \item Prove that this problem is NP-hard by showing it is exactly as a graph partioning problem with constraints;
          \item List and discuss promising heuristics; 
          \item Extend the model to compute the partitioning algorithm cost in the total application execution -- the idea is to fold this into the workload $\mathcal{W}^{*} = (\mathcal{A}^{*}, G)$ submitted to $p_0$, where $\mathcal{A}^{*}$ is a particular graph partitioning algorithm;
          \item Characterize Equation~\ref{eq:throughput_p} with some experiments on a CPU/GPU environment.
        \end{enumerate}}


\section{Related Work}
\label{sec:related}

This project spans the following areas: graph algorithms, in general, and parallel graph algorithms, in particular; graph algorithms on GPUs; and parallel graph processing frameworks. This section briefly comments on the related literature collected so far.

{\bf Parallel graph algorithms.} Although graph algorithms are a well studied area, leveraging parallel architecture to accelerate the algorithm runtime is far from straightforward task. A parallel implementation may require substantial changes to the original sequential algorithm and strong assumptions about the parallel platform. The required changes and assumptions come at the expense of making the algorithm less portable across platforms.

Although the PRAM~\cite{Fortune78} parallel processing model is vastly used in the study of parallel algorithms~\cite{Quinn1984,Atallah1984}, some recent works try to provide performance analysis on distributed memory machines models. Meyer and Sanders~\cite{Meyer2003}, for example, gives a parallel algorithm for single source shortest path named $\Delta$-stepping. The authors analyze the performance of the algorithm in the PRAM model and provide extensions to distributed memory model. The algorithm works by keeping nodes with tentative distances in separate buckets where each bucket represent the distances within the range of size $\Delta$. In each phase the algorithm consider the nodes in the non-empty buckets and edges of weight up to $\Delta$. Parallelism is achieved by processing the nodes in a given bucket concurrently. 

{\bf Graph algorithms on GPUs.} Previous work investigates the use of GPUs to accelerate graph computation~\cite{Harish2007, Katz2008, Sungpack2010, dehne2010exploring}. For example, Sungpack et al. ~\cite{Sungpack2010} shows that, in the case of breadth-first search, a single Nvidia Tesla GPU offers up to 2.5x speedup compared to dual socket, quad-core Intel Xeon symmetric multiprocessor on realistic workloads. However, those works focus on few graph algorithms, mainly breadth-first search, single source shortest path, and all pairs shortest path. Moreover, the implementations assume that the graph fits GPU memory, which puts a limitation on the size of graphs that can be processed. 

A natural scaling path is to run large graph algorithms into multi-GPU systems. Katz et al. ~\cite{Katz2008} implement a hand-crafted version of all pairs shortest path algorithm for multi-GPU systems. However, the proposed approach is tightly coupled with the problem and cannot be generalized to other algorithms. Additionally, the implementation provided by Katz et al. assumes an adjacency matrix graph representation that imposes a significant space overhead for sparse graphs, a characteristic that is common in the majority of real-world graphs. Yet another, nevertheless fundamental, challenge to harness multi-GPU systems is to adapt these algorithms from the PRAM model to a distributed memory machines model such as BSP or LogP. Gerbessiotis and Valiant provide results on the performance impact on emulating a DMM model with PRAM~\cite{Gerbessiotis92}. 

{\bf Graph processing frameworks.} The Parallel BGL~\cite{gregor2005parallel} and Pregel~\cite{Malewicz2009} present frameworks to implement distributed graph algorithms. Both frameworks specify generic concepts for designing distributed graph algorithms. The frameworks are based on abstractions that are common among graph algorithms. Examples of such abstractions include vertices, edges, property maps {\em (key-value pairs)}, and mechanisms for data propagation and graph traversal. 

It is worth noting that PBGL, Pregel, and others~\cite{Zhao2009} assume traditional cluster systems (i.e., nodes connected via network and containing traditional multi-core processors). This work targets, on the other hand, GPU-based systems, which have different computational model and offer different trade-offs and challenges. 

To the best of our knowledge, there is no graph processing framework optimized for hybrid GPU-based platforms. A framework specialized for GPU-based platforms could offer optimizations to minimize communication overhead over the GPU's high-latency I/O channels (e.g., employing compact messaging and graph representations to minimize communication and space footprint); also it could offer graph-specific abstractions that hide the GPU's complex memory model while efficiently leveraging it (e.g., transparently utilizing shared memory). 


\section{Concluding Remarks}
\label{sec:conclusion}



%% References
\bibliographystyle{abbrv}
\bibliography{graphs}
\balancecolumns
\end{document}
